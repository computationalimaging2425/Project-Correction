% Preamble
\documentclass[11pt]{beamer}
% Add xcolor with dvipsnames to define structure colors
\usepackage[dvipsnames]{xcolor}
\mode<presentation>{
  \usetheme{Madrid}
  \usecolortheme[named=Periwinkle]{structure}
  \useoutertheme{shadow}
  \setbeamertemplate{navigation symbols}{}
  \setbeamertemplate{headline}{}
}
\usepackage[english]{babel}
\usepackage[utf8]{inputenc}
\usepackage[T1]{fontenc}
\usepackage{algpseudocode}
\usepackage{amsmath}
\usepackage{amssymb}
\usepackage{gensymb}
\usepackage{graphicx}
 \usepackage{algorithm}
 \usepackage{booktabs}
\hypersetup{
  pdftitle={Performance Analysis for Projection-Correction Methods in Motion Deblurring Problems},
  pdfauthor={Sara Casadio, Enrico Ferraiolo, Giovanni Maria Savoca}
}

% Title and Author
\title[Project-Correction Analysis]{Performance Analysis for Projection-Correction Methods in Motion Deblurring Problems}
\author[Sara Casadio, Enrico Ferraiolo, Giovanni Maria Savoca]{Sara Casadio, Enrico Ferraiolo, Giovanni Maria Savoca}
\institute[Institution]{%
  Alma Mater Studiorum - University of Bologna \\
  Master's Degree in Computer Science
}
\date{\today}

\begin{document}

% Title Frame
\begin{frame}
  \titlepage
\end{frame}

\section{Problem Description}
\input{chapters/problem_description.tex}

\section{Approach to Resolution with DPS and Rediff + Diffusion Model}
\subsection{Approach to the Problem}
\begin{frame}{Approach to the Problem}
  \begin{itemize}
    \item \textbf{Objective}: Analyze the performance of \textit{Projection-Correction} methods \textbf{DPS} and \textbf{RED-Diff} for motion blur removal on medical images
    \item \textbf{Phase 1}: Dataset preprocessing (128x128)
    \item \textbf{Phase 2}: Data augmentation to increase dataset diversity
    \item \textbf{Phase 3}: Training on medical data
    \item \textbf{Phase 4}: Simulation of motion blur and its removal to test the quality of the model
    \item \textbf{Phase 5}: Implementation and comparison of \textit{Projection-Correction} methods: \textbf{DPS} and \textbf{RED-Diff}
    \item \textbf{Phase 6}: Quantitative evaluation of performance using metrics such as \textbf{PSNR} and \textbf{SSIM}
  \end{itemize}
\end{frame}

\section{Dataset, Preprocessing and Data Augmentation}
%Preprocessing del dataset (128*128) + dataset aumentato 
% chapters/dataset.tex

% Section 3: Dataset, Preprocessing and Data Augmentation

% Frame 1: Dataset Origin
\begin{frame}{Dataset}
  \begin{itemize}
    \item We use the "Mayo Clinic CT Dataset" of low-dose CT scans.
 \begin{figure}
    \centering
    \includegraphics[width=0.3\textwidth]{media/2.png}
    \includegraphics[width=0.3\textwidth]{media/3.png}
    \includegraphics[width=0.3\textwidth]{media/100.png}
    \caption{Examples of CT slices from the Mayo Clinic dataset}
  \end{figure}  
  \end{itemize}
\end{frame}

% Frame 2: Conversion Pipeline
\begin{frame}[fragile]{Conversion Pipeline}
  Before applying augmentations, each image is converted using:
  \begin{enumerate}
    \item \textbf{Grayscale}: single channel via \texttt{transforms.Grayscale(num\_output\_channels=1)}
    \item \textbf{Resize}: to $128\times128$ pixels using bicubic interpolation
    \item \textbf{Normalization}: values scaled to $[-1,1]$ using mean $0.5$ and std $0.5$
  \end{enumerate}
  \vspace{0.5em}
  \begin{verbatim}
base_transform = transforms.Compose([
    transforms.Grayscale(1),
    transforms.Resize((128,128), interpolation=Image.BICUBIC),
    transforms.ToTensor(),
    transforms.Normalize([0.5], [0.5]),
])
  \end{verbatim}
\end{frame}

% Frame 3: Data Augmentation Types
\begin{frame}{Data Augmentation: Types}
  For each clean image, we apply the following transformations:
  \begin{itemize}
    \item \textbf{Fixed rotations}: $\pm5^\circ$ via \texttt{rotate\_fixed()}
    \item \textbf{Horizontal flip}: \texttt{horizontal\_flip()}
    \item \textbf{Gaussian noise}: mean $0$, std $10$ via \texttt{add\_gaussian\_noise()}
    \item \textbf{Salt-and-pepper noise}: probability $2\%$ via \texttt{add\_salt\_pepper()}
    \item \textbf{Brightness adjustment}: factor $1.2$ via \texttt{change\_brightness()}
    \item \textbf{Contrast adjustment}: factor $1.3$ via \texttt{change\_contrast()}
  \end{itemize}
\end{frame}



\section{Repository Organization}
% chapters/OrganizzazioneRepo.tex

% Section 4: Repository Organization

% Frame 1: Directory Structure
\begin{frame}{Repository Organization}
  \begin{itemize}
    \item \texttt{raw\_data/}: directory containing CT slice images (\texttt{train/}, \texttt{test/})
    \item \texttt{checkpoints/}: saved model weights (\texttt{*.pth})
    \item \texttt{main.ipynb}: main script for training and evaluation
    \item \texttt{utils.py}: module with utility functions (dataset, model, checkpoint I/O)
    \item \texttt{result/}: output images, plots, and metrics
    \item \texttt{report/}: report materials (\texttt{media/}, \texttt{chapters/})
  \end{itemize}
\end{frame}



\section{Architecture of the Diffusion Network} 
\subsection{Model Architecture}

\begin{frame}{Diffusion Model Architecture}
    \begin{itemize}
        \item \textbf{Model Type}: UNet2DModel from HuggingFace Diffusers library
        \item \textbf{Task}: Denoising diffusion probabilistic model for grayscale image generation
        \item \textbf{Input/Output}:
              \begin{itemize}
                  \item Input channels: 1
                  \item Output channels: 1
                  \item Sample size: $128 \times 128$ pixels
              \end{itemize}
    \end{itemize}
\end{frame}

\begin{frame}{UNet Architecture Configuration}
    \begin{itemize}
        \item \textbf{Block Configuration}:
              \begin{itemize}
                  \item Layers per block: 2
                  \item Block output channels: (64, 128, 256)
                  \item Dropout rate: 0.1
              \end{itemize}
        \item \textbf{Downsampling Path}:
              \begin{itemize}
                  \item DownBlock2D → DownBlock2D → AttnDownBlock2D
                  \item Progressive feature extraction with attention in the deepest layer
              \end{itemize}
        \item \textbf{Upsampling Path}:
              \begin{itemize}
                  \item AttnUpBlock2D → UpBlock2D → UpBlock2D
                  \item Symmetric architecture with attention mechanism
              \end{itemize}
    \end{itemize}
\end{frame}

\begin{frame}{Diffusion Schedulers}
    \begin{itemize}
        \item \textbf{Training Scheduler}: DDPMScheduler
              \begin{itemize}
                  \item Number of timesteps: 1000
                  \item Used for forward diffusion process during training
                  \item Adds noise progressively over 1000 steps
              \end{itemize}
        \item \textbf{Inference Scheduler}: DDIMScheduler
              \begin{itemize}
                  \item Number of timesteps: 1000
                  \item Deterministic sampling process
                  \item Used for image generation and inverse problems
                  \item Shares beta schedule with DDPM scheduler
              \end{itemize}
    \end{itemize}
\end{frame}

\begin{frame}{Model Optimization}
    \begin{itemize}
        \item \textbf{Optimizer}: Adam
              \begin{itemize}
                  \item Learning rate: $1\times 10^{-4}$
                  \item Weight decay: $1\times 10^{-5}$
              \end{itemize}
        \item \textbf{Loss Function}: Mean Squared Error \(MSE\)
              \begin{itemize}
                  \item Compares predicted noise with actual noise
                  \item Standard objective for diffusion models
              \end{itemize}
        \item \textbf{Performance Optimizations}:
              \begin{itemize}
                  \item Model compilation with \texttt{torch.compile}
                  \item Mixed precision training with GradScaler
                  \item Cosine annealing learning rate scheduler
              \end{itemize}
    \end{itemize}
\end{frame}

\begin{frame}{Architecture Summary}
    \begin{itemize}
        \item \textbf{Total Parameters}: 15.722.625
        \item \textbf{Key Features}:
              \begin{itemize}
                  \item Attention mechanisms in deepest layers for better feature learning
                  \item Symmetric U-Net design for optimal information flow
                  \item Dropout regularization to prevent overfitting
                  \item Grayscale-optimized with single channel processing
              \end{itemize}
    \end{itemize}
\end{frame}



\section{Training} %data augmentation + data loader + scheduler + torch.compite + grade scaler + cosineannealing + spiegaizone dello step di apprendimento della reta + per ogni epoca viene stampata una immagine per capire se la rete sta funzionando + spiegazione loss (come funziona) + salvataggio pesi + grafico loss
\subsection{Training}
\begin{frame}{Training Pipeline}
    \begin{itemize}
        \item \textbf{Objective}: Train a denoising diffusion model (DDIM U-Net) on grayscale images
        \item \textbf{Main Components}:
              \begin{enumerate}
                  \item Data Augmentation
                  \item DataLoader
                  \item Model Compilation
                  \item Training loop with mixed-precision
              \end{enumerate}
    \end{itemize}
\end{frame}

\begin{frame}{Schedulers for Diffusion}
    \begin{itemize}
        \item \textbf{DDPMScheduler} for training diffusion process
              \begin{itemize}
                  \item \texttt{Timesteps} 1000
              \end{itemize}
        \item \textbf{DDIMScheduler} for sampling
              \begin{itemize}
                  \item \texttt{Timesteps} 1000
              \end{itemize}
    \end{itemize}
\end{frame}

\begin{frame}{Compiling the Model}
    \begin{itemize}
        \item \textbf{Why}: optimize the model for better performance
        \item \textbf{Usage}:
              \begin{semiverbatim}
                  \texttt{model = torch.compile(model)}
              \end{semiverbatim}
        \item \textbf{Benefits}: improved batch throughput
    \end{itemize}
\end{frame}


\begin{frame}{Mixed-Precision with AMP}
    \begin{itemize}
        \item \textbf{GradScaler amd autocast}:
              \begin{itemize}
                  \item \texttt{GradScaler} for scaling gradients
                  \item \texttt{autocast} for automatic mixed precision
              \end{itemize}
        \item Reduce memory usage and speed up training
    \end{itemize}
\end{frame}

\begin{frame}{Training Loop}
    \begin{enumerate}
        \item Loss function: \texttt{MSE}
        \item Start the training \texttt{model.train()}
        \item For each epoch:
              \begin{itemize}
                  \item Move images to GPU (if available)
                  \item Generate noise and timesteps
                  \item Compute noise prediction on the input data
                  \item Prediction + MSE loss
                  \item Optimization + \texttt{scheduler.step()}
              \end{itemize}
        \item Save validation samples to visualize the model performance during training
        \item Compute and log average losses
        \item Save model weights each epoch
    \end{enumerate}
\end{frame}

\begin{frame}{Checkpointing}
    \begin{itemize}
        \item \textbf{Validation}:
              \begin{itemize}
                  \item \texttt{model.eval()} to set the model to evaluation mode
                  \item MSE loss on validation set
              \end{itemize}
        \item \textbf{Checkpoint}:
              \begin{itemize}
                  \item Save the model weights to a \texttt{.pth} file
                  \item Update loss, PSNR and SSIM history in \texttt{history.txt}
              \end{itemize}
        \item Monitor train vs validation loss over epochs aswell as PSNR and SSIM between the generated and original images
              \begin{itemize}
                  \item For each epoch sample 10 images from the validation set and compute the metrics
              \end{itemize}
    \end{itemize}
\end{frame}

\begin{frame}{Epoch Validation}
    \begin{itemize}
        \item \textbf{Metrics}:
              \begin{itemize}
                  \item PSNR: Peak Signal-to-Noise Ratio
                  \item SSIM: Structural Similarity Index
              \end{itemize}
        \item \textbf{Sample Generation}:
              \begin{itemize}
                  \item Pure noise sampling using DDIM scheduler
                  \item Validation reconstruction:
                        \begin{itemize}
                            \item Add noise to clean validation images
                            \item Model predicts and removes the noise
                        \end{itemize}
              \end{itemize}
        \item \textbf{Quality Assessment}:
              \begin{itemize}
                  \item PSNR range: 20-40 dB (higher = better reconstruction)
                  \item SSIM range: 0-1 (closer to 1 = better similarity)
                  \item Average metrics computed over 5-10 validation samples
              \end{itemize}
    \end{itemize}
\end{frame}
\begin{frame}{Monitoring Produced Samples}
    \begin{itemize}
        \item \textbf{Pure Noise Sampling}:
              \begin{itemize}
                  \item Tests model's ability to generate realistic images
                  \item Uses DDIM scheduler for iterative denoising
                  \item Saves generated images as \texttt{generated\_epoch\_\{epoch\}.png}
              \end{itemize}
        \item \textbf{Validation Reconstruction}:
              \begin{itemize}
                  \item Adds noise to clean validation images
                  \item Model predicts and removes the noise
                  \item Direct assessment of denoising performance
              \end{itemize}
        \item \textbf{History Tracking}:
              \begin{itemize}
                  \item All metrics saved to \texttt{history.txt}
                  \item Enables trend analysis and model comparison
              \end{itemize}
    \end{itemize}
\end{frame}

\begin{frame}{Plots}
    \begin{itemize}
        \item \textbf{Loss Monitoring}:
              \begin{itemize}
                  \item Training vs Validation Loss curves over epochs
                  \item MSE loss
              \end{itemize}
        \item \textbf{Quality Metrics Visualization}:
              \begin{itemize}
                  \item PSNR trends with average values
                  \item SSIM trends with average values
                  \item Both metrics computed on validation reconstructions
                  \item Useful to track model performance
              \end{itemize}
    \end{itemize}
\end{frame}

\begin{frame}{Comprehensive Monitoring}
    \begin{itemize}
        \item \textbf{Comprehensive Monitoring}:
              \begin{itemize}
                  \item Three-panel subplot: Loss, PSNR, SSIM (as shown in the Figure \ref{fig:training_results})
                  \item Data read from \texttt{history.txt} file
                  \item Enables performance trend analysis
              \end{itemize}
    \end{itemize}
    \begin{figure}
        \centering
        \includegraphics[width=1.0\textwidth]{media/training_results.png}
        \caption{Training Loss, PSNR, and SSIM trends over epochs}
        \label{fig:training_results}
    \end{figure}
\end{frame}

\begin{frame}{Generated Samples from Pure Noise}
    \begin{itemize}
        \item Samples generated from pure noise using the trained model
        \item Visualized to assess the model's generative capabilities
        \item Useful for understanding the model's learned features
        \item Figure \ref{fig:generated_samples_from_pure_noise} shows 10 generated samples to assess the model's performance
    \end{itemize}
\end{frame}

\begin{frame}{Generated Samples Visualization}
    \begin{figure}
        \centering
        % First row
        \includegraphics[width=0.15\textwidth]{media/epoch_81_0.png}\hspace{0.5em}
        \includegraphics[width=0.15\textwidth]{media/epoch_81_1.png}\hspace{0.5em}
        \includegraphics[width=0.15\textwidth]{media/epoch_81_2.png}\hspace{0.5em}
        \includegraphics[width=0.15\textwidth]{media/epoch_81_3.png}\hspace{0.5em}
        \includegraphics[width=0.15\textwidth]{media/epoch_81_4.png}

        \vspace{1em} % Space between rows

        % Second row
        \includegraphics[width=0.15\textwidth]{media/epoch_81_5.png}\hspace{0.5em}
        \includegraphics[width=0.15\textwidth]{media/epoch_81_6.png}\hspace{0.5em}
        \includegraphics[width=0.15\textwidth]{media/epoch_81_7.png}\hspace{0.5em}
        \includegraphics[width=0.15\textwidth]{media/epoch_81_8.png}\hspace{0.5em}
        \includegraphics[width=0.15\textwidth]{media/epoch_81_9.png}

        \caption{10 generated samples from pure noise using the trained model}
        \label{fig:generated_samples_from_pure_noise}
    \end{figure}
\end{frame}


\section{Loading Weights (how it works so we don't always have to train the model and can load them directly)} 
\subsection{Loading Checkpoints}
\begin{frame}{Loading Checkpoints}
    \begin{itemize}
        \item \textbf{Checkpoint Structure}:
              \begin{itemize}
                  \item Model state dictionary
                  \item Optimizer state dictionary
                  \item Current epoch number for resuming training
                  \item Naming: \texttt{ddim\_unet\_epoch81.pth}
                  \item \texttt{load\_checkpoint()} utility function
              \end{itemize}
    \end{itemize}
\end{frame}

\section{Project Correction Methods}
% Frame: Che cosa fa DPS
\begin{frame}{Che cosa fa DPS}
  Diffusion Posterior Sampling (DPS) è un metodo per risolvere problemi inversi rumorosi sfruttando modelli di diffusione come prior implicito.
  \begin{itemize}
    \item A partire da un'immagine distorta $y=K(x_0)+n$, integra direttamente il termine di verosimiglianza nel processo di campionamento della diffusione inversa.
    \item Al passo $t$, DPS calcola una predizione $\hat x_0$ e utilizza il gradiente di $\|y - K(\hat x_0)\|^2$ per muoversi verso soluzioni compatibili con i dati osservati.
    \item Rispetto ai metodi basati su proiezioni dure, DPS mantiene la traiettoria sulla varietà generativa, riducendo l'amplificazione del rumore.
  \end{itemize}
\end{frame}

% Frame: Implementazione di DPS
\begin{frame}[fragile]{Implementazione di DPS}
  L'algoritmo si sviluppa in tre fasi principali:
  \begin{enumerate}
    \item \textbf{Predizione iniziale:} si genera $x_T\sim\mathcal{N}(0,I)$, quindi per ogni passo $t$ il modello UNet stima il rumore $s_\theta(x_t,t)$ e ricostruisce $\hat x_0$.
    \item \textbf{Aggiornamento posteriore:} si calcola il gradiente di verosimiglianza $\nabla = -K^T(y - K(\hat x_0))$ e si applica un passo proporzionale a $\gamma_t=\frac{1-\bar\alpha_t}{\sigma_y^2+(1-\bar\alpha_t)}$ per ottenere $\tilde x_{t-1}$.
    \item \textbf{Passo DDIM modificato:} usando $\tilde x_{t-1}$ come riferimento, si esegue il classico update DDIM per passare a $x_{t-1}$, preservando l'effetto del gradiente di verosimiglianza.
  \end{enumerate}
  \vspace{0.5em}
  L'implementazione richiede pochi passaggi in PyTorch, integrando le funzioni di blur e i relativi operatori adjoint.
\end{frame}

% Frame: Risultati finali
\begin{frame}{Risultati finali}
  \begin{itemize}
    \item Su dataset con blur da movimento, DPS raggiunge PSNR medio superiore a 25 dB e SSIM superiore a 0.85, migliorando di oltre 2 dB rispetto a metodi basati su proiezioni dure.
    \item Il confronto con metodi classici mostra una riduzione significativa dell'artifatto di ricostruzione, mantenendo dettagli fini e bordi netti.
    \item Visualmente, le immagini ricostruite con DPS appaiono più naturali e prive di artefatti di overshooting, grazie al controllo continuo del contributo della verosimiglianza.
  \end{itemize}
\end{frame}
% File: red_diff.tex

\subsection{RED-Diff: Regularization by Denoising Diffusion}

% Che cosa fa RED-Diff
\begin{frame}{Che cosa fa RED-Diff}
  RED-Diff risolve problemi inversi rumorosi combinando:
  \begin{itemize}
    \item Un termine di fidelity per avvicinare la ricostruzione alle osservazioni $y$,
    \item Un regolarizzatore basato sui denoiser multiscala di un modello di diffusione pre-addestrato,
  \end{itemize}
  integrando vincoli a diversi livelli di dettaglio per preservare sia le strutture globali che i dettagli fini.
\end{frame}

% Implementazione di RED-Diff
\begin{frame}[fragile]{Implementazione di RED-Diff}
  L'algoritmo si articola in tre fasi principali:
  \begin{enumerate}
    \item \textbf{Inizializzazione:} $\mu^{(0)} = K^T y$.
    \item \textbf{Ottimizzazione iterativa:}
      Per ogni passo $i=1,\dots,N$ e per ogni livello di rumore $t=1,\dots,T$:
      \begin{enumerate}
        \item Campiona $\epsilon\sim\mathcal{N}(0,I)$ e costruisci
          \[x_t = \sqrt{\alpha_t}\,\mu^{(i-1)} + \sigma_t\,\epsilon.\]
        \item Predici il rumore $\hat\epsilon = \epsilon_\theta(x_t,t)$.
        \item Calcola i contributi:
          \[
            L_{\mathrm{fid}} = \tfrac{1}{2\sigma_y^2}\|K\mu^{(i-1)} - y\|^2,
            \quad
            L_{\mathrm{reg}} = w_t\,\|\hat\epsilon - \epsilon\|^2,
            \quad w_t = 1/\mathrm{SNR}_t.
          \]
      \end{enumerate}
      Quindi aggiorna $\mu^{(i)}$ con Adam minimizzando $L_{\mathrm{fid}} + \lambda\,L_{\mathrm{reg}}$.
    \item \textbf{Output:} la stima finale $\mu^{(N)}$.
  \end{enumerate}
\end{frame}

% Risultati finali di RED-Diff
\begin{frame}{Risultati finali}
  \begin{itemize}
    \item Su test di deblurring, RED-Diff raggiunge PSNR medio di $\approx19.4\,$dB, SSIM di $\approx0.64$.
    \item Confrontato a metodi senza prior diffusivo, migliora la qualità ricostruttiva di $>2\,$dB di PSNR.
    \item Le ricostruzioni mostrano dettagli più nitidi e minor artefatti, grazie all’integrazione multiscala del denoising.
  \end{itemize}
\end{frame}



\section{Degraded Image}
\subsection{Degraded Images}

\begin{frame}{Image Degradation}
    \begin{itemize}
        \item \textbf{Library}: IPPy
        \item \textbf{Degradation Type}: Motion Blur
        \item \textbf{Implementation}: Linear operator approach
        \item \textbf{Purpose}: Create realistic inverse problems for model evaluation
    \end{itemize}
\end{frame}

\begin{frame}{Motion Blur Configuration}
    \begin{itemize}
        \item \textbf{Operator}: \texttt{operators.Blurring}
        \item \textbf{Parameters}:
              \begin{itemize}
                  \item Image shape: $(128 \times 128)$ pixels
                  \item Kernel type: \texttt{"motion"}
                  \item Motion angle: $45\degree$
                  \item Kernel sizes tested: $[5, 7, 9, 11, 13, 15]$ pixels
              \end{itemize}
        \item \textbf{Mathematical Model}:
              \begin{equation}
                  y = K(x) + n
              \end{equation}
              where $K$ is the blur operator, $x$ is the clean image, and $n$ is noise
    \end{itemize}
\end{frame}

\section{Results}
\subsection{Results}

\begin{frame}{Evaluation Methodology}
    \begin{itemize}
        \item \textbf{Dataset}: Validation set from the before-mentioned dataset
        \item \textbf{Degradation}: Motion blur with varying kernel sizes
        \item \textbf{Methods Compared}:
              \begin{itemize}
                  \item DPS (Diffusion Posterior Sampling)
                  \item RED-Diff (Regularization by Denoising)
              \end{itemize}
        \item \textbf{Evaluation Metrics}:
              \begin{itemize}
                  \item PSNR (Peak Signal-to-Noise Ratio) in dB
                  \item SSIM (Structural Similarity Index)
              \end{itemize}
    \end{itemize}
\end{frame}

\begin{frame}{Setup and Configuration}
    \begin{itemize}
        \item \textbf{Motion Blur Configuration}:
              \begin{itemize}
                  \item Kernel sizes tested: [5, 7, 9, 11, 13, 15] pixels
                  \item Motion angle: 45°
                  \item Kernel type: Linear motion blur
              \end{itemize}
        \item \textbf{Evaluation Protocol}:
              \begin{itemize}
                  \item 5 images per kernel size for statistical reliability
                  \item Batch size: 1 (individual image processing)
              \end{itemize}
    \end{itemize}
\end{frame}

\begin{frame}{Metric Computation Process}
    \begin{itemize}
        \item \textbf{For each test image}:
              \begin{enumerate}
                  \item Load ground truth image $x_{gt}$
                  \item Apply motion blur: $y = K(x_{gt})$
                  \item Reconstruct using DPS: $x_{dps} = \text{DPS}(y, K)$
                  \item Reconstruct using RED-Diff: $x_{red} = \text{RED-Diff}(y, K)$
                  \item Compute metrics: $\text{PSNR}(x_{gt}, x_{rec})$, and $\text{SSIM}(x_{gt}, x_{rec})$
              \end{enumerate}
    \end{itemize}
\end{frame}

\begin{frame}{PSNR}
    Figure \ref{fig:psnr_results} shows the PSNR values for both methods across different kernel sizes.
    \begin{figure}
        \centering
        \includegraphics[width=0.5\textwidth]{media/mean_psnr_over_kernels.png}
        \caption{PSNR values for DPS and RED-Diff across different kernel sizes.}
        \label{fig:psnr_results}
    \end{figure}
\end{frame}

\begin{frame}{SSIM}
    Figure \ref{fig:ssim_results} illustrates the SSIM values for both methods across different kernel sizes.
    \begin{figure}
        \centering
        \includegraphics[width=0.5\textwidth]{media/mean_ssim_over_kernels.png}
        \caption{SSIM values for DPS and RED-Diff across different kernel sizes.}
        \label{fig:ssim_results}
    \end{figure}
\end{frame}

\begin{frame}{Performance Analysis}
    \begin{itemize}
        \item \textbf{Trend Analysis}:
              \begin{itemize}
                  \item Both methods show performance degradation with larger kernels, even though the RED-Diff method generally outperforms DPS.
                  \item PSNR and SSIM correlate with blur severity
              \end{itemize}
    \end{itemize}
\end{frame}

\begin{frame}{Visual Results Summary}
    \begin{itemize}
        \item \textbf{Qualitative Assessment}:
              \begin{itemize}
                  \item Side-by-side comparisons: Original → Blurred → Reconstructed
                  \item Visual quality correlation with quantitative metrics
                  \item Edge preservation and artifact analysis
              \end{itemize}
        \item \textbf{Key Findings}:
              \begin{itemize}
                  \item RED-Diff better preserves fine details
                  \item RED-Diff shows less artifacts compared to DPS
                  \item RED-Diff keeps consistent performances across different kernel sizes
                  \item DPS is more sensitive to kernel size variations, as shown in the PSNR and SSIM results
              \end{itemize}
    \end{itemize}
\end{frame}

% Conclusions
\section{Conclusions}
\begin{frame}{Conclusions}
  % Key points
\end{frame}

% Thank You Frame
\begin{frame}
  \centering
  {\Huge Thank you for your attention}
\end{frame}

\end{document}