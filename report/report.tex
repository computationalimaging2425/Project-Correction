% Preamble
\documentclass[11pt]{beamer}
% Add xcolor with dvipsnames to define structure colors
\usepackage[dvipsnames]{xcolor}
\mode<presentation>{
  \usetheme{Madrid}
  \usecolortheme[named=Periwinkle]{structure}
  \useoutertheme{shadow}
  \setbeamertemplate{navigation symbols}{}
  \setbeamertemplate{headline}{}
}
\usepackage[italian]{babel}
\usepackage[utf8]{inputenc}
\usepackage[T1]{fontenc}
\usepackage{graphicx}
\hypersetup{
  pdftitle={Performance Analysis for Projection-Correction Methods in Motion Deblurring Problems},
  pdfauthor={Sara Casadio, Enrico Ferraiolo, Giovanni Savoca}
}

% Title and Author
\title[Progetto]{Performance Analysis for Projection-Correction Methods in Motion Deblurring Problems}
\author[Autore]{Sara Casadio, Enrico Ferraiolo, Giovanni Savoca}
\institute[Istituzione]{%
  Alma Mater Studiorum - Università di Bologna \\
  Corso di Laurea in Informatica
}
\date{\today}

\begin{document}

% Title Frame
\begin{frame}
  \titlepage
\end{frame}

% Section 1
\section{Descrizione del problema}
\subsection{Sottosezione 1.1}
\begin{frame}{Sottosezione 1.1}
  % Contenuto qui
\end{frame}

% Section 2
\section{Approccio di risoluzione con dps e rediff + modello diffusivo}
\input{capitoli/ApproccioRisoluzione.tex}

% Section 3
\section{Dataset} %Preprocessing del dataset (128*128) + dataset aumentato 
% chapters/dataset.tex

% Section 3: Dataset, Preprocessing and Data Augmentation

% Frame 1: Dataset Origin
\begin{frame}{Dataset Origin}
  \begin{itemize}
    \item We use the "Mayo Clinic CT Dataset" of low-dose CT scans, available via the link provided in this report.
    \item It contains a total of 6,400 2D slices in PNG format, extracted from 20 different patients.
    \item The images are organized into:
      \begin{itemize}
        \item \texttt{raw\_data/train/}: 5,120 slices for training (80\% of the dataset)
        \item \texttt{raw\_data/test/}: 1,280 slices for testing (20\% of the dataset)
      \end{itemize}
 \begin{figure}
    \centering
    \includegraphics[width=0.3\textwidth]{media/2.png}
    \includegraphics[width=0.3\textwidth]{media/3.png}
    \includegraphics[width=0.3\textwidth]{media/100.png}
    \caption{Examples of CT slices from the Mayo Clinic dataset}
  \end{figure}  
  \end{itemize}
\end{frame}

% Frame 2: Conversion Pipeline
\begin{frame}[fragile]{Conversion Pipeline}
  Before applying augmentations, each image is converted using:
  \begin{enumerate}
    \item \textbf{Grayscale}: single channel via \texttt{transforms.Grayscale(num\_output\_channels=1)}
    \item \textbf{Resize}: to $128\times128$ pixels using bicubic interpolation
    \item \textbf{Normalization}: values scaled to $[-1,1]$ using mean $0.5$ and std $0.5$
  \end{enumerate}
  \vspace{0.5em}
  \begin{verbatim}
base_transform = transforms.Compose([
    transforms.Grayscale(1),
    transforms.Resize((128,128), interpolation=Image.BICUBIC),
    transforms.ToTensor(),
    transforms.Normalize([0.5], [0.5]),
])
  \end{verbatim}
\end{frame}

% Frame 3: Data Augmentation Types
\begin{frame}{Data Augmentation: Types}
  For each clean image, we apply the following transformations:
  \begin{itemize}
    \item \textbf{Fixed rotations}: $\pm5^\circ$ via \texttt{rotate\_fixed()}
    \item \textbf{Horizontal flip}: \texttt{horizontal\_flip()}
    \item \textbf{Gaussian noise}: mean $0$, std $10$ via \texttt{add\_gaussian\_noise()}
    \item \textbf{Salt-and-pepper noise}: probability $2\%$ via \texttt{add\_salt\_pepper()}
    \item \textbf{Brightness adjustment}: factor $1.2$ via \texttt{change\_brightness()}
    \item \textbf{Contrast adjustment}: factor $1.3$ via \texttt{change\_contrast()}
  \end{itemize}
\end{frame}


% Section 4
\section{organizzazione della repository}
\subsection{Sottosezione 2.1}
\begin{frame}{Sottosezione 2.1}
  % Contenuto qui
\end{frame}

% Section 5
\section{architettura della rete diffusiva}
\subsection{Sottosezione 2.1}
\begin{frame}{Sottosezione 2.1}
  % Contenuto qui
\end{frame}

% Section 6
\section{Training} %data augmentation + data loader + scheduler + torch.compite + grade scaler + cosineannealing + spiegaizone dello step di apprendimento della reta + per ogni epoca viene stampata una immagine per capire se la rete sta funzionando + spiegazione loss (come funziona) + salvataggio pesi + grafico loss
\subsection{Training}
\begin{frame}{Training Pipeline}
    \begin{itemize}
        \item \textbf{Objective}: Train a denoising diffusion model (DDIM U-Net) on grayscale images
        \item \textbf{Main Components}:
              \begin{enumerate}
                  \item Data Augmentation
                  \item DataLoader
                  \item Model Compilation
                  \item Training loop with mixed-precision
              \end{enumerate}
    \end{itemize}
\end{frame}

\begin{frame}{Data Augmentation}
    \begin{itemize}
        \item \textbf{Base Dataset}: Dataset Mayo
              \begin{itemize}
                  \item Grayscale → 1 channel
                  \item Resize images to \texttt{128 $\times$ 128}
              \end{itemize}
        \item \textbf{Augmentations} (8 types):
              \begin{itemize}
                  \item \emph{None}: no transformation
                  \item \texttt{Rotation} ±5° (rotation + centering)
                  \item \texttt{Flip} horizontal
                  \item \texttt{Gaussian noise} (mean=0, std=10)
                  \item \texttt{Salt and pepper noise} (prob=2\%)
                  \item \texttt{Brightness} (factor=1.2)
                  \item \texttt{Contrast} (factor=1.3)
              \end{itemize}
    \end{itemize}
\end{frame}

\begin{frame}{Schedulers for Diffusion}
    \begin{itemize}
        \item \textbf{DDPMScheduler} for training diffusion process
              \begin{itemize}
                  \item \texttt{Timesteps} 1000
              \end{itemize}
        \item \textbf{DDIMScheduler} for sampling
              \begin{itemize}
                  \item \texttt{Timesteps} 1000
              \end{itemize}
    \end{itemize}
\end{frame}

\begin{frame}{Compiling the Model}
    \begin{itemize}
        \item \textbf{Why}: optimize the model for better performance
        \item \textbf{Usage}:
              \begin{semiverbatim}
                  \texttt{model = torch.compile(model)}
              \end{semiverbatim}
        \item \textbf{Benefits}: improved batch throughput
    \end{itemize}
\end{frame}


\begin{frame}{Mixed-Precision with AMP}
    \begin{itemize}
        \item \textbf{GradScaler amd autocast}:
              \begin{itemize}
                  \item \texttt{GradScaler} for scaling gradients
                  \item \texttt{autocast} for automatic mixed precision
              \end{itemize}
        \item Reduce memory usage and speed up training
    \end{itemize}
\end{frame}

\begin{frame}{Training Loop}
    \begin{enumerate}
        \item Loss function: \texttt{MSE}
        \item Start the training \texttt{model.train()}
        \item For each epoch:
              \begin{itemize}
                  \item Move images to GPU (if available)
                  \item Generate noise and timesteps
                  \item Compute noise prediction on the input data
                  \item Prediction + MSE loss
                  \item Optimization + \texttt{scheduler.step()}
              \end{itemize}
        \item Save validation samples to visualize the model performance during training
        \item Compute and log average losses
        \item Save model weights each epoch
    \end{enumerate}
\end{frame}

\begin{frame}{Checkpointing}
    \begin{itemize}
        \item \textbf{Validation}:
              \begin{itemize}
                  \item \texttt{model.eval()} to set the model to evaluation mode
                  \item MSE loss on validation set
              \end{itemize}
        \item \textbf{Checkpoint}:
              \begin{itemize}
                  \item Save the model weights to a \texttt{.pth} file
                  \item Update loss, PSNR and SSIM history in \texttt{history.txt}
              \end{itemize}
        \item Monitor train vs validation loss over epochs aswell as PSNR and SSIM between the generated and original images
              \begin{itemize}
                  \item For each epoch sample 10 images from the validation set and compute the metrics
              \end{itemize}
    \end{itemize}
\end{frame}

\begin{frame}{Epoch Validation}
    \begin{itemize}
        \item \textbf{Metrics}:
              \begin{itemize}
                  \item PSNR: Peak Signal-to-Noise Ratio
                  \item SSIM: Structural Similarity Index
              \end{itemize}
        \item \textbf{Sample Generation}:
              \begin{itemize}
                  \item Pure noise sampling using DDIM scheduler
                  \item Validation reconstruction:
                        \begin{itemize}
                            \item Add noise to clean validation images
                            \item Model predicts and removes the noise
                        \end{itemize}
              \end{itemize}
        \item \textbf{Quality Assessment}:
              \begin{itemize}
                  \item PSNR range: 20-40 dB (higher = better reconstruction)
                  \item SSIM range: 0-1 (closer to 1 = better similarity)
                  \item Average metrics computed over 5-10 validation samples
              \end{itemize}
    \end{itemize}
\end{frame}
\begin{frame}{Monitoring Produced Samples}
    \begin{itemize}
        \item \textbf{Pure Noise Sampling}:
              \begin{itemize}
                  \item Tests model's ability to generate realistic images
                  \item Uses DDIM scheduler for iterative denoising
                  \item Saves generated images as \texttt{generated\_epoch\_\{epoch\}.png}
              \end{itemize}
        \item \textbf{Validation Reconstruction}:
              \begin{itemize}
                  \item Adds noise to clean validation images
                  \item Model predicts and removes the noise
                  \item Direct assessment of denoising performance
              \end{itemize}
        \item \textbf{History Tracking}:
              \begin{itemize}
                  \item All metrics saved to \texttt{history.txt}
                  \item Enables trend analysis and model comparison
              \end{itemize}
    \end{itemize}
\end{frame}

\begin{frame}{Plots}
    \begin{itemize}
        \item \textbf{Loss Monitoring}:
              \begin{itemize}
                  \item Training vs Validation Loss curves over epochs
                  \item MSE loss
              \end{itemize}
        \item \textbf{Quality Metrics Visualization}:
              \begin{itemize}
                  \item PSNR trends with average values
                  \item SSIM trends with average values
                  \item Both metrics computed on validation reconstructions
                  \item Useful to track model performance
              \end{itemize}
    \end{itemize}
\end{frame}

\begin{frame}{Comprehensive Monitoring}
    \begin{itemize}
        \item \textbf{Comprehensive Monitoring}:
              \begin{itemize}
                  \item Three-panel subplot: Loss, PSNR, SSIM (as shown in the Figure \ref{fig:training_results})
                  \item Data read from \texttt{history.txt} file
                  \item Enables performance trend analysis
              \end{itemize}
    \end{itemize}
    \begin{figure}
        \centering
        \includegraphics[width=1.0\textwidth]{media/training_results.png}
        \caption{Training Loss, PSNR, and SSIM trends over epochs}
        \label{fig:training_results}
    \end{figure}
\end{frame}

% Section 7
\section{Caricamento dei pesi (come funziona che non dobbiamo sempre allenare il modello e possiamo caricarli deìirettamente )}
\subsection{Sottosezione 2.1}
\begin{frame}{Sottosezione 2.1}
  % Contenuto qui
\end{frame}

% Section 8
\section{codice dei due metodi}
\subsection{Sottosezione 2.1}
\begin{frame}{Sottosezione 2.1}
  % Contenuto qui
\end{frame}

% Section 9
\section{codice immagine degradata}
\subsection{Sottosezione 2.1}
\begin{frame}{Sottosezione 2.1}
  % Contenuto qui
\end{frame}

% Section 9
\section{risultati}
\subsection{Sottosezione 2.1}
\begin{frame}{Sottosezione 2.1}
  % Contenuto qui
\end{frame}

% Conclusions
\section{Conclusioni}
\begin{frame}{Conclusioni}
  % Punti chiave
\end{frame}

% Thank You Frame
\begin{frame}
  \centering
  {\Huge Grazie per l'attenzione}
\end{frame}

\end{document}