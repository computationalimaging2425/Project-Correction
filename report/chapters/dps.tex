\subsection{DPS: Diffusion Posterior Sampling}
\begin{frame}{DPS: Diffusion Posterior Sampling}
  \begin{itemize}
    \item \textbf{DPS} is a method for solving inverse problems using pre-trained diffusion models
    \item Key idea: combine data fidelity with diffusion model prior during reverse sampling
    \item Modifies the standard DDIM reverse process to incorporate measurement consistency
  \end{itemize}
\end{frame}


\begin{frame}{DPS - Overview}
  \textbf{Algorithm Overview:}
  \begin{enumerate}
    \item Start with noisy initialization $x_T \sim \mathcal{N}(0, I)$
    \item For each timestep $t$: predict $x_0$ using UNet
    \item Apply posterior correction: $x_0^{post} = x_0^{pred} + \gamma_t \cdot K^T(y - K x_0^{pred})$
    \item Continue DDIM step with corrected $x_0^{post}$
  \end{enumerate}

  \textbf{Posterior Correction Weight:}
  $$\gamma_t = \frac{\sigma_{prior}^2}{\sigma_y^2 + \sigma_{prior}^2}$$
  where $\sigma_{prior}^2 = 1 - \alpha_t$ and $\sigma_y$ is measurement noise
\end{frame}

% Frame: Implementation of DPS
\begin{frame}{Implementation of DPS}
  \textbf{Algorithm Steps:}
  \begin{enumerate}
    \item Initialize $x_t$ with random noise
    \item Predict noise $\epsilon_\theta(x_t, t)$ using UNet
    \item Compute $x_0^{pred} = \frac{x_t - \sqrt{1-\alpha_t} \epsilon_\theta}{\sqrt{\alpha_t}}$
    \item Apply DPS correction with gradient step
    \item Update to next timestep using DDIM
  \end{enumerate}
\end{frame}
