\subsection{Training}
\begin{frame}{Training Pipeline}
    \begin{itemize}
        \item \textbf{Objective}: Train a denoising diffusion model (DDIM U-Net) on grayscale images
        \item \textbf{Main Components}:
              \begin{enumerate}
                  \item Data Augmentation
                  \item DataLoader
                  \item Model Compilation
                  \item Training loop with mixed-precision
              \end{enumerate}
    \end{itemize}
\end{frame}

\begin{frame}{Data Augmentation}
    \begin{itemize}
        \item \textbf{Base Dataset}: Dataset Mayo
              \begin{itemize}
                  \item Grayscale → 1 channel
                  \item Resize images to \texttt{128 $\times$ 128}
              \end{itemize}
        \item \textbf{Augmentations} (8 types):
              \begin{itemize}
                  \item \emph{None}: no transformation
                  \item \texttt{Rotation} ±5° (rotation + centering)
                  \item \texttt{Flip} horizontal
                  \item \texttt{Gaussian noise} (mean=0, std=10)
                  \item \texttt{Salt and pepper noise} (prob=2\%)
                  \item \texttt{Brightness} (factor=1.2)
                  \item \texttt{Contrast} (factor=1.3)
              \end{itemize}
        \item \textbf{Implementazione essenziale}:
    \end{itemize}
\end{frame}

\begin{frame}{Schedulers for Diffusion}
    \begin{itemize}
        \item \textbf{DDPMScheduler} for training diffusion process
              \begin{itemize}
                  \item \texttt{Timesteps} 1000
              \end{itemize}
        \item \textbf{DDIMScheduler} for sampling
              \begin{itemize}
                  \item \texttt{Timesteps} 1000
              \end{itemize}
    \end{itemize}
\end{frame}

\begin{frame}{Compiling the Model}
    \begin{itemize}
        \item \textbf{Why}: optimize the model for better performance
        \item \textbf{Usage}:
              \begin{semiverbatim}
                  \texttt{model = torch.compile(model)}
              \end{semiverbatim}
        \item \textbf{Benefits}: improved batch throughput
    \end{itemize}
\end{frame}


\begin{frame}{Mixed-Precision with AMP}
    \begin{itemize}
        \item \textbf{GradScaler amd autocast}:
              \begin{itemize}
                  \item \texttt{GradScaler} for scaling gradients
                  \item \texttt{autocast} for automatic mixed precision
              \end{itemize}
        \item Reduce memory usage and speed up training
    \end{itemize}
\end{frame}

\begin{frame}{Training Loop}
    \begin{enumerate}
        \item Loss function: \texttt{MSE}
        \item Start the training \texttt{model.train()}
        \item For each epoch:
              \begin{itemize}
                  \item Move images to GPU (if available)
                  \item Generate noise and timesteps
                  \item Compute noise prediction on the input data
                  \item Prediction + MSE loss
                  \item Optimization + \texttt{scheduler.step()}
              \end{itemize}
        \item Save validation samples to visualize the model performance during training
        \item Compute and log average losses
        \item Save model weights each epoch
    \end{enumerate}
\end{frame}

\begin{frame}{Validation and Checkpointing}
    \begin{itemize}
        \item \textbf{Validation}:
              \begin{itemize}
                  \item \texttt{model.eval()} to set the model to evaluation mode
                  \item MSE loss on validation set
              \end{itemize}
        \item \textbf{Checkpoint}:
              \begin{itemize}
                  \item Save the model weights to a \texttt{.pth} file
                  \item Update loss, PSNR and SSIM history in \texttt{history.txt}
              \end{itemize}
        \item Monitor train vs validation loss over epochs aswell as PSNR and SSIM between the generated and original images
                \begin{itemize}
                    \item For each epoch sample 10 images from the validation set and compute the metrics
                \end{itemize}
    \end{itemize}
\end{frame}

\begin{frame}{Loss Plot}
    \begin{center}
        \begin{itemize}
            \item \textbf{Loss Plot}: visualizes the training and validation loss over epochs
            \item \textbf{Purpose}:
                  \begin{itemize}
                      \item Monitor the model's performance
                  \end{itemize}
        \end{itemize}
    \end{center}
\end{frame}
