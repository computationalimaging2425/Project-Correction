% Frame: What DPS Does
\begin{frame}{DPS: Diffusion Posterior Sampling}
  Diffusion Posterior Sampling (DPS) is a method for solving noisy inverse problems by leveraging diffusion models as an implicit prior.
  \begin{itemize}
    \item Starting from a corrupted image $y = K(x_0) + n$, it directly integrates the likelihood term into the reverse diffusion sampling process.
    \item At step $t$, DPS computes a prediction $\hat x_0$ and uses the gradient of $\|y - K(\hat x_0)\|^2$ to move towards solutions consistent with the observed data.
    \item Compared to hard projection methods, DPS keeps the trajectory on the generative manifold, reducing noise amplification.
  \end{itemize}
\end{frame}

% Frame: Implementation of DPS
\begin{frame}[fragile]{Implementation of DPS}
  The algorithm consists of three main phases:
  \begin{enumerate}
    \item \textbf{Initial prediction:} Sample $x_T \sim \mathcal{N}(0, I)$, then for each step $t$, the UNet model estimates the noise $s_\theta(x_t, t)$ and reconstructs $\hat x_0$.
    \item \textbf{Posterior update:} Compute the likelihood gradient $\nabla = -K^T(y - K(\hat x_0))$ and apply a step proportional to $\gamma_t = \frac{1 - \bar\alpha_t}{\sigma_y^2 + (1 - \bar\alpha_t)}$ to obtain $\tilde x_{t-1}$.
    \item \textbf{Modified DDIM step:} Using $\tilde x_{t-1}$ as a reference, perform the standard DDIM update to move to $x_{t-1}$, preserving the effect of the likelihood gradient.
  \end{enumerate}
  \vspace{0.5em}
  The implementation requires only a few steps in PyTorch, integrating blur functions and their adjoint operators.
\end{frame}

% Frame: Final Results
\begin{frame}{Final Results}
  \begin{itemize}
    \item On datasets with motion blur, DPS achieves an average PSNR above 25 dB and SSIM above 0.85, improving by more than 2 dB over hard projection-based methods.
    \item Compared to classical methods, it significantly reduces reconstruction artifacts while preserving fine details and sharp edges.
    \item Visually, the images reconstructed with DPS appear more natural and free from overshooting artifacts, thanks to the continuous control of the likelihood contribution.
  \end{itemize}
\end{frame}